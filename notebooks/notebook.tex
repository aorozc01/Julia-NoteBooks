
% Default to the notebook output style

    


% Inherit from the specified cell style.




    
\documentclass[11pt]{article}

    
    
    \usepackage[T1]{fontenc}
    % Nicer default font (+ math font) than Computer Modern for most use cases
    \usepackage{mathpazo}

    % Basic figure setup, for now with no caption control since it's done
    % automatically by Pandoc (which extracts ![](path) syntax from Markdown).
    \usepackage{graphicx}
    % We will generate all images so they have a width \maxwidth. This means
    % that they will get their normal width if they fit onto the page, but
    % are scaled down if they would overflow the margins.
    \makeatletter
    \def\maxwidth{\ifdim\Gin@nat@width>\linewidth\linewidth
    \else\Gin@nat@width\fi}
    \makeatother
    \let\Oldincludegraphics\includegraphics
    % Set max figure width to be 80% of text width, for now hardcoded.
    \renewcommand{\includegraphics}[1]{\Oldincludegraphics[width=.8\maxwidth]{#1}}
    % Ensure that by default, figures have no caption (until we provide a
    % proper Figure object with a Caption API and a way to capture that
    % in the conversion process - todo).
    \usepackage{caption}
    \DeclareCaptionLabelFormat{nolabel}{}
    \captionsetup{labelformat=nolabel}

    \usepackage{adjustbox} % Used to constrain images to a maximum size 
    \usepackage{xcolor} % Allow colors to be defined
    \usepackage{enumerate} % Needed for markdown enumerations to work
    \usepackage{geometry} % Used to adjust the document margins
    \usepackage{amsmath} % Equations
    \usepackage{amssymb} % Equations
    \usepackage{textcomp} % defines textquotesingle
    % Hack from http://tex.stackexchange.com/a/47451/13684:
    \AtBeginDocument{%
        \def\PYZsq{\textquotesingle}% Upright quotes in Pygmentized code
    }
    \usepackage{upquote} % Upright quotes for verbatim code
    \usepackage{eurosym} % defines \euro
    \usepackage[mathletters]{ucs} % Extended unicode (utf-8) support
    \usepackage[utf8x]{inputenc} % Allow utf-8 characters in the tex document
    \usepackage{fancyvrb} % verbatim replacement that allows latex
    \usepackage{grffile} % extends the file name processing of package graphics 
                         % to support a larger range 
    % The hyperref package gives us a pdf with properly built
    % internal navigation ('pdf bookmarks' for the table of contents,
    % internal cross-reference links, web links for URLs, etc.)
    \usepackage{hyperref}
    \usepackage{longtable} % longtable support required by pandoc >1.10
    \usepackage{booktabs}  % table support for pandoc > 1.12.2
    \usepackage[inline]{enumitem} % IRkernel/repr support (it uses the enumerate* environment)
    \usepackage[normalem]{ulem} % ulem is needed to support strikethroughs (\sout)
                                % normalem makes italics be italics, not underlines
    

    
    
    % Colors for the hyperref package
    \definecolor{urlcolor}{rgb}{0,.145,.698}
    \definecolor{linkcolor}{rgb}{.71,0.21,0.01}
    \definecolor{citecolor}{rgb}{.12,.54,.11}

    % ANSI colors
    \definecolor{ansi-black}{HTML}{3E424D}
    \definecolor{ansi-black-intense}{HTML}{282C36}
    \definecolor{ansi-red}{HTML}{E75C58}
    \definecolor{ansi-red-intense}{HTML}{B22B31}
    \definecolor{ansi-green}{HTML}{00A250}
    \definecolor{ansi-green-intense}{HTML}{007427}
    \definecolor{ansi-yellow}{HTML}{DDB62B}
    \definecolor{ansi-yellow-intense}{HTML}{B27D12}
    \definecolor{ansi-blue}{HTML}{208FFB}
    \definecolor{ansi-blue-intense}{HTML}{0065CA}
    \definecolor{ansi-magenta}{HTML}{D160C4}
    \definecolor{ansi-magenta-intense}{HTML}{A03196}
    \definecolor{ansi-cyan}{HTML}{60C6C8}
    \definecolor{ansi-cyan-intense}{HTML}{258F8F}
    \definecolor{ansi-white}{HTML}{C5C1B4}
    \definecolor{ansi-white-intense}{HTML}{A1A6B2}

    % commands and environments needed by pandoc snippets
    % extracted from the output of `pandoc -s`
    \providecommand{\tightlist}{%
      \setlength{\itemsep}{0pt}\setlength{\parskip}{0pt}}
    \DefineVerbatimEnvironment{Highlighting}{Verbatim}{commandchars=\\\{\}}
    % Add ',fontsize=\small' for more characters per line
    \newenvironment{Shaded}{}{}
    \newcommand{\KeywordTok}[1]{\textcolor[rgb]{0.00,0.44,0.13}{\textbf{{#1}}}}
    \newcommand{\DataTypeTok}[1]{\textcolor[rgb]{0.56,0.13,0.00}{{#1}}}
    \newcommand{\DecValTok}[1]{\textcolor[rgb]{0.25,0.63,0.44}{{#1}}}
    \newcommand{\BaseNTok}[1]{\textcolor[rgb]{0.25,0.63,0.44}{{#1}}}
    \newcommand{\FloatTok}[1]{\textcolor[rgb]{0.25,0.63,0.44}{{#1}}}
    \newcommand{\CharTok}[1]{\textcolor[rgb]{0.25,0.44,0.63}{{#1}}}
    \newcommand{\StringTok}[1]{\textcolor[rgb]{0.25,0.44,0.63}{{#1}}}
    \newcommand{\CommentTok}[1]{\textcolor[rgb]{0.38,0.63,0.69}{\textit{{#1}}}}
    \newcommand{\OtherTok}[1]{\textcolor[rgb]{0.00,0.44,0.13}{{#1}}}
    \newcommand{\AlertTok}[1]{\textcolor[rgb]{1.00,0.00,0.00}{\textbf{{#1}}}}
    \newcommand{\FunctionTok}[1]{\textcolor[rgb]{0.02,0.16,0.49}{{#1}}}
    \newcommand{\RegionMarkerTok}[1]{{#1}}
    \newcommand{\ErrorTok}[1]{\textcolor[rgb]{1.00,0.00,0.00}{\textbf{{#1}}}}
    \newcommand{\NormalTok}[1]{{#1}}
    
    % Additional commands for more recent versions of Pandoc
    \newcommand{\ConstantTok}[1]{\textcolor[rgb]{0.53,0.00,0.00}{{#1}}}
    \newcommand{\SpecialCharTok}[1]{\textcolor[rgb]{0.25,0.44,0.63}{{#1}}}
    \newcommand{\VerbatimStringTok}[1]{\textcolor[rgb]{0.25,0.44,0.63}{{#1}}}
    \newcommand{\SpecialStringTok}[1]{\textcolor[rgb]{0.73,0.40,0.53}{{#1}}}
    \newcommand{\ImportTok}[1]{{#1}}
    \newcommand{\DocumentationTok}[1]{\textcolor[rgb]{0.73,0.13,0.13}{\textit{{#1}}}}
    \newcommand{\AnnotationTok}[1]{\textcolor[rgb]{0.38,0.63,0.69}{\textbf{\textit{{#1}}}}}
    \newcommand{\CommentVarTok}[1]{\textcolor[rgb]{0.38,0.63,0.69}{\textbf{\textit{{#1}}}}}
    \newcommand{\VariableTok}[1]{\textcolor[rgb]{0.10,0.09,0.49}{{#1}}}
    \newcommand{\ControlFlowTok}[1]{\textcolor[rgb]{0.00,0.44,0.13}{\textbf{{#1}}}}
    \newcommand{\OperatorTok}[1]{\textcolor[rgb]{0.40,0.40,0.40}{{#1}}}
    \newcommand{\BuiltInTok}[1]{{#1}}
    \newcommand{\ExtensionTok}[1]{{#1}}
    \newcommand{\PreprocessorTok}[1]{\textcolor[rgb]{0.74,0.48,0.00}{{#1}}}
    \newcommand{\AttributeTok}[1]{\textcolor[rgb]{0.49,0.56,0.16}{{#1}}}
    \newcommand{\InformationTok}[1]{\textcolor[rgb]{0.38,0.63,0.69}{\textbf{\textit{{#1}}}}}
    \newcommand{\WarningTok}[1]{\textcolor[rgb]{0.38,0.63,0.69}{\textbf{\textit{{#1}}}}}
    
    
    % Define a nice break command that doesn't care if a line doesn't already
    % exist.
    \def\br{\hspace*{\fill} \\* }
    % Math Jax compatability definitions
    \def\gt{>}
    \def\lt{<}
    % Document parameters
    \title{DipoleForce\_SDK}
    
    
    

    % Pygments definitions
    
\makeatletter
\def\PY@reset{\let\PY@it=\relax \let\PY@bf=\relax%
    \let\PY@ul=\relax \let\PY@tc=\relax%
    \let\PY@bc=\relax \let\PY@ff=\relax}
\def\PY@tok#1{\csname PY@tok@#1\endcsname}
\def\PY@toks#1+{\ifx\relax#1\empty\else%
    \PY@tok{#1}\expandafter\PY@toks\fi}
\def\PY@do#1{\PY@bc{\PY@tc{\PY@ul{%
    \PY@it{\PY@bf{\PY@ff{#1}}}}}}}
\def\PY#1#2{\PY@reset\PY@toks#1+\relax+\PY@do{#2}}

\expandafter\def\csname PY@tok@w\endcsname{\def\PY@tc##1{\textcolor[rgb]{0.73,0.73,0.73}{##1}}}
\expandafter\def\csname PY@tok@c\endcsname{\let\PY@it=\textit\def\PY@tc##1{\textcolor[rgb]{0.25,0.50,0.50}{##1}}}
\expandafter\def\csname PY@tok@cp\endcsname{\def\PY@tc##1{\textcolor[rgb]{0.74,0.48,0.00}{##1}}}
\expandafter\def\csname PY@tok@k\endcsname{\let\PY@bf=\textbf\def\PY@tc##1{\textcolor[rgb]{0.00,0.50,0.00}{##1}}}
\expandafter\def\csname PY@tok@kp\endcsname{\def\PY@tc##1{\textcolor[rgb]{0.00,0.50,0.00}{##1}}}
\expandafter\def\csname PY@tok@kt\endcsname{\def\PY@tc##1{\textcolor[rgb]{0.69,0.00,0.25}{##1}}}
\expandafter\def\csname PY@tok@o\endcsname{\def\PY@tc##1{\textcolor[rgb]{0.40,0.40,0.40}{##1}}}
\expandafter\def\csname PY@tok@ow\endcsname{\let\PY@bf=\textbf\def\PY@tc##1{\textcolor[rgb]{0.67,0.13,1.00}{##1}}}
\expandafter\def\csname PY@tok@nb\endcsname{\def\PY@tc##1{\textcolor[rgb]{0.00,0.50,0.00}{##1}}}
\expandafter\def\csname PY@tok@nf\endcsname{\def\PY@tc##1{\textcolor[rgb]{0.00,0.00,1.00}{##1}}}
\expandafter\def\csname PY@tok@nc\endcsname{\let\PY@bf=\textbf\def\PY@tc##1{\textcolor[rgb]{0.00,0.00,1.00}{##1}}}
\expandafter\def\csname PY@tok@nn\endcsname{\let\PY@bf=\textbf\def\PY@tc##1{\textcolor[rgb]{0.00,0.00,1.00}{##1}}}
\expandafter\def\csname PY@tok@ne\endcsname{\let\PY@bf=\textbf\def\PY@tc##1{\textcolor[rgb]{0.82,0.25,0.23}{##1}}}
\expandafter\def\csname PY@tok@nv\endcsname{\def\PY@tc##1{\textcolor[rgb]{0.10,0.09,0.49}{##1}}}
\expandafter\def\csname PY@tok@no\endcsname{\def\PY@tc##1{\textcolor[rgb]{0.53,0.00,0.00}{##1}}}
\expandafter\def\csname PY@tok@nl\endcsname{\def\PY@tc##1{\textcolor[rgb]{0.63,0.63,0.00}{##1}}}
\expandafter\def\csname PY@tok@ni\endcsname{\let\PY@bf=\textbf\def\PY@tc##1{\textcolor[rgb]{0.60,0.60,0.60}{##1}}}
\expandafter\def\csname PY@tok@na\endcsname{\def\PY@tc##1{\textcolor[rgb]{0.49,0.56,0.16}{##1}}}
\expandafter\def\csname PY@tok@nt\endcsname{\let\PY@bf=\textbf\def\PY@tc##1{\textcolor[rgb]{0.00,0.50,0.00}{##1}}}
\expandafter\def\csname PY@tok@nd\endcsname{\def\PY@tc##1{\textcolor[rgb]{0.67,0.13,1.00}{##1}}}
\expandafter\def\csname PY@tok@s\endcsname{\def\PY@tc##1{\textcolor[rgb]{0.73,0.13,0.13}{##1}}}
\expandafter\def\csname PY@tok@sd\endcsname{\let\PY@it=\textit\def\PY@tc##1{\textcolor[rgb]{0.73,0.13,0.13}{##1}}}
\expandafter\def\csname PY@tok@si\endcsname{\let\PY@bf=\textbf\def\PY@tc##1{\textcolor[rgb]{0.73,0.40,0.53}{##1}}}
\expandafter\def\csname PY@tok@se\endcsname{\let\PY@bf=\textbf\def\PY@tc##1{\textcolor[rgb]{0.73,0.40,0.13}{##1}}}
\expandafter\def\csname PY@tok@sr\endcsname{\def\PY@tc##1{\textcolor[rgb]{0.73,0.40,0.53}{##1}}}
\expandafter\def\csname PY@tok@ss\endcsname{\def\PY@tc##1{\textcolor[rgb]{0.10,0.09,0.49}{##1}}}
\expandafter\def\csname PY@tok@sx\endcsname{\def\PY@tc##1{\textcolor[rgb]{0.00,0.50,0.00}{##1}}}
\expandafter\def\csname PY@tok@m\endcsname{\def\PY@tc##1{\textcolor[rgb]{0.40,0.40,0.40}{##1}}}
\expandafter\def\csname PY@tok@gh\endcsname{\let\PY@bf=\textbf\def\PY@tc##1{\textcolor[rgb]{0.00,0.00,0.50}{##1}}}
\expandafter\def\csname PY@tok@gu\endcsname{\let\PY@bf=\textbf\def\PY@tc##1{\textcolor[rgb]{0.50,0.00,0.50}{##1}}}
\expandafter\def\csname PY@tok@gd\endcsname{\def\PY@tc##1{\textcolor[rgb]{0.63,0.00,0.00}{##1}}}
\expandafter\def\csname PY@tok@gi\endcsname{\def\PY@tc##1{\textcolor[rgb]{0.00,0.63,0.00}{##1}}}
\expandafter\def\csname PY@tok@gr\endcsname{\def\PY@tc##1{\textcolor[rgb]{1.00,0.00,0.00}{##1}}}
\expandafter\def\csname PY@tok@ge\endcsname{\let\PY@it=\textit}
\expandafter\def\csname PY@tok@gs\endcsname{\let\PY@bf=\textbf}
\expandafter\def\csname PY@tok@gp\endcsname{\let\PY@bf=\textbf\def\PY@tc##1{\textcolor[rgb]{0.00,0.00,0.50}{##1}}}
\expandafter\def\csname PY@tok@go\endcsname{\def\PY@tc##1{\textcolor[rgb]{0.53,0.53,0.53}{##1}}}
\expandafter\def\csname PY@tok@gt\endcsname{\def\PY@tc##1{\textcolor[rgb]{0.00,0.27,0.87}{##1}}}
\expandafter\def\csname PY@tok@err\endcsname{\def\PY@bc##1{\setlength{\fboxsep}{0pt}\fcolorbox[rgb]{1.00,0.00,0.00}{1,1,1}{\strut ##1}}}
\expandafter\def\csname PY@tok@kc\endcsname{\let\PY@bf=\textbf\def\PY@tc##1{\textcolor[rgb]{0.00,0.50,0.00}{##1}}}
\expandafter\def\csname PY@tok@kd\endcsname{\let\PY@bf=\textbf\def\PY@tc##1{\textcolor[rgb]{0.00,0.50,0.00}{##1}}}
\expandafter\def\csname PY@tok@kn\endcsname{\let\PY@bf=\textbf\def\PY@tc##1{\textcolor[rgb]{0.00,0.50,0.00}{##1}}}
\expandafter\def\csname PY@tok@kr\endcsname{\let\PY@bf=\textbf\def\PY@tc##1{\textcolor[rgb]{0.00,0.50,0.00}{##1}}}
\expandafter\def\csname PY@tok@bp\endcsname{\def\PY@tc##1{\textcolor[rgb]{0.00,0.50,0.00}{##1}}}
\expandafter\def\csname PY@tok@fm\endcsname{\def\PY@tc##1{\textcolor[rgb]{0.00,0.00,1.00}{##1}}}
\expandafter\def\csname PY@tok@vc\endcsname{\def\PY@tc##1{\textcolor[rgb]{0.10,0.09,0.49}{##1}}}
\expandafter\def\csname PY@tok@vg\endcsname{\def\PY@tc##1{\textcolor[rgb]{0.10,0.09,0.49}{##1}}}
\expandafter\def\csname PY@tok@vi\endcsname{\def\PY@tc##1{\textcolor[rgb]{0.10,0.09,0.49}{##1}}}
\expandafter\def\csname PY@tok@vm\endcsname{\def\PY@tc##1{\textcolor[rgb]{0.10,0.09,0.49}{##1}}}
\expandafter\def\csname PY@tok@sa\endcsname{\def\PY@tc##1{\textcolor[rgb]{0.73,0.13,0.13}{##1}}}
\expandafter\def\csname PY@tok@sb\endcsname{\def\PY@tc##1{\textcolor[rgb]{0.73,0.13,0.13}{##1}}}
\expandafter\def\csname PY@tok@sc\endcsname{\def\PY@tc##1{\textcolor[rgb]{0.73,0.13,0.13}{##1}}}
\expandafter\def\csname PY@tok@dl\endcsname{\def\PY@tc##1{\textcolor[rgb]{0.73,0.13,0.13}{##1}}}
\expandafter\def\csname PY@tok@s2\endcsname{\def\PY@tc##1{\textcolor[rgb]{0.73,0.13,0.13}{##1}}}
\expandafter\def\csname PY@tok@sh\endcsname{\def\PY@tc##1{\textcolor[rgb]{0.73,0.13,0.13}{##1}}}
\expandafter\def\csname PY@tok@s1\endcsname{\def\PY@tc##1{\textcolor[rgb]{0.73,0.13,0.13}{##1}}}
\expandafter\def\csname PY@tok@mb\endcsname{\def\PY@tc##1{\textcolor[rgb]{0.40,0.40,0.40}{##1}}}
\expandafter\def\csname PY@tok@mf\endcsname{\def\PY@tc##1{\textcolor[rgb]{0.40,0.40,0.40}{##1}}}
\expandafter\def\csname PY@tok@mh\endcsname{\def\PY@tc##1{\textcolor[rgb]{0.40,0.40,0.40}{##1}}}
\expandafter\def\csname PY@tok@mi\endcsname{\def\PY@tc##1{\textcolor[rgb]{0.40,0.40,0.40}{##1}}}
\expandafter\def\csname PY@tok@il\endcsname{\def\PY@tc##1{\textcolor[rgb]{0.40,0.40,0.40}{##1}}}
\expandafter\def\csname PY@tok@mo\endcsname{\def\PY@tc##1{\textcolor[rgb]{0.40,0.40,0.40}{##1}}}
\expandafter\def\csname PY@tok@ch\endcsname{\let\PY@it=\textit\def\PY@tc##1{\textcolor[rgb]{0.25,0.50,0.50}{##1}}}
\expandafter\def\csname PY@tok@cm\endcsname{\let\PY@it=\textit\def\PY@tc##1{\textcolor[rgb]{0.25,0.50,0.50}{##1}}}
\expandafter\def\csname PY@tok@cpf\endcsname{\let\PY@it=\textit\def\PY@tc##1{\textcolor[rgb]{0.25,0.50,0.50}{##1}}}
\expandafter\def\csname PY@tok@c1\endcsname{\let\PY@it=\textit\def\PY@tc##1{\textcolor[rgb]{0.25,0.50,0.50}{##1}}}
\expandafter\def\csname PY@tok@cs\endcsname{\let\PY@it=\textit\def\PY@tc##1{\textcolor[rgb]{0.25,0.50,0.50}{##1}}}

\def\PYZbs{\char`\\}
\def\PYZus{\char`\_}
\def\PYZob{\char`\{}
\def\PYZcb{\char`\}}
\def\PYZca{\char`\^}
\def\PYZam{\char`\&}
\def\PYZlt{\char`\<}
\def\PYZgt{\char`\>}
\def\PYZsh{\char`\#}
\def\PYZpc{\char`\%}
\def\PYZdl{\char`\$}
\def\PYZhy{\char`\-}
\def\PYZsq{\char`\'}
\def\PYZdq{\char`\"}
\def\PYZti{\char`\~}
% for compatibility with earlier versions
\def\PYZat{@}
\def\PYZlb{[}
\def\PYZrb{]}
\makeatother


    % Exact colors from NB
    \definecolor{incolor}{rgb}{0.0, 0.0, 0.5}
    \definecolor{outcolor}{rgb}{0.545, 0.0, 0.0}



    
    % Prevent overflowing lines due to hard-to-break entities
    \sloppy 
    % Setup hyperref package
    \hypersetup{
      breaklinks=true,  % so long urls are correctly broken across lines
      colorlinks=true,
      urlcolor=urlcolor,
      linkcolor=linkcolor,
      citecolor=citecolor,
      }
    % Slightly bigger margins than the latex defaults
    
    \geometry{verbose,tmargin=1in,bmargin=1in,lmargin=1in,rmargin=1in}
    
    

    \begin{document}
    
    
    \maketitle
    
    

    
    \begin{Verbatim}[commandchars=\\\{\}]
{\color{incolor}In [{\color{incolor}308}]:} \PY{k}{using} \PY{n}{PyPlot}
          \PY{k}{using} \PY{n}{Unitful}
\end{Verbatim}


    Here I explore the parameters required to create a SDK interferometer
using the dipole force. The sequence goes like this

\begin{enumerate}
\def\labelenumi{\arabic{enumi}.}
\tightlist
\item
  Prepare the atom in a ground state
\item
  apply a microwave \(\pi/2\) pulse to prepare atom in a superposition
\item
  apply a pulse with a focused beam such that the atom experiences the
  maximum gradient

  \begin{itemize}
  \tightlist
  \item
    this will kick the atom in a direction that will depend on the state
  \end{itemize}
\item
  wait a time T
\item
  apply a microwave \(\pi\) pulse to swap atomic states
\item
  apply a pulse with a focused beam such that the atom experiences a
  field gradient

  \begin{itemize}
  \tightlist
  \item
    this will redirect the wavepackets by kicking the atom in an
    opposite direction as in line 3 above
  \end{itemize}
\item
  wait a time T
\item
  apply a microwave \(\pi/2\) pulse to create a superposition
\item
  apply a pulse with a focused beam such that the atom experiences a
  field gradient

  \begin{itemize}
  \tightlist
  \item
    this will be a kick in oppositie direction for each state
  \end{itemize}
\end{enumerate}

    \begin{figure}
\centering
\includegraphics{imgs/SDK_FieldGradient/AIDiagram.jpg}
\caption{title}
\end{figure}

    The dipole force on an atom is equal to the gradient of the dipole
potential by the following equation \(\vec{F} = -\vec{\Delta}U(r,z)\).
The dipole potential can be written in terms of the atomic
polarizability \(\alpha\),
\(U = -\frac{1}{2\epsilon_0c}Re(\alpha) I(r)\).

The acceleration the atom experiences can be calculated using Newton's
second law, \(\vec{a}=\vec{F}/m\).

    \subsubsection{Define Constants}\label{define-constants}

    \begin{Verbatim}[commandchars=\\\{\}]
{\color{incolor}In [{\color{incolor}177}]:} \PY{n}{c} \PY{o}{=} \PY{l+m+mf}{3e8}\PY{n}{u}\PY{l+s}{\PYZdq{}}\PY{l+s}{m}\PY{l+s}{/}\PY{l+s}{s}\PY{l+s}{\PYZdq{}}\PY{p}{;} \PY{c}{\PYZsh{} m/s}
          \PY{n}{ϵ0} \PY{o}{=} \PY{l+m+mf}{8.85e\PYZhy{}12}\PY{n}{u}\PY{l+s}{\PYZdq{}}\PY{l+s}{C}\PY{l+s}{/}\PY{l+s}{(}\PY{l+s}{V}\PY{l+s}{*}\PY{l+s}{m}\PY{l+s}{)}\PY{l+s}{\PYZdq{}}\PY{p}{;} \PY{c}{\PYZsh{} C/(V*m)}
          \PY{n}{amu} \PY{o}{=} \PY{l+m+mf}{1.67e\PYZhy{}27}\PY{n}{u}\PY{l+s}{\PYZdq{}}\PY{l+s}{k}\PY{l+s}{g}\PY{l+s}{\PYZdq{}}\PY{p}{;} \PY{c}{\PYZsh{} kg}
          \PY{n}{g} \PY{o}{=} \PY{l+m+mi}{10}\PY{n}{u}\PY{l+s}{\PYZdq{}}\PY{l+s}{m}\PY{l+s}{/}\PY{l+s}{s}\PY{l+s}{\PYZca{}}\PY{l+s}{2}\PY{l+s}{\PYZdq{}}\PY{p}{;}
          \PY{n}{kB} \PY{o}{=} \PY{l+m+mf}{1.38e\PYZhy{}23}\PY{n}{u}\PY{l+s}{\PYZdq{}}\PY{l+s}{J}\PY{l+s}{/}\PY{l+s}{K}\PY{l+s}{\PYZdq{}}\PY{p}{;}
\end{Verbatim}


    \subsubsection{Atom Characteristics}\label{atom-characteristics}

    \begin{Verbatim}[commandchars=\\\{\}]
{\color{incolor}In [{\color{incolor}125}]:} \PY{n}{Γ} \PY{o}{=} \PY{l+m+mi}{2}\PY{o}{*}\PY{n+nb}{π}\PY{o}{*}\PY{l+m+mf}{4.575e6}\PY{n}{u}\PY{l+s}{\PYZdq{}}\PY{l+s}{H}\PY{l+s}{z}\PY{l+s}{\PYZdq{}}\PY{p}{;} \PY{c}{\PYZsh{} Hz}
          \PY{n}{Isat} \PY{o}{=} \PY{l+m+mf}{2.5}\PY{n}{u}\PY{l+s}{\PYZdq{}}\PY{l+s}{m}\PY{l+s}{W}\PY{l+s}{/}\PY{l+s}{c}\PY{l+s}{m}\PY{l+s}{\PYZca{}}\PY{l+s}{2}\PY{l+s}{\PYZdq{}}\PY{p}{;} \PY{c}{\PYZsh{} mW/cm\PYZca{}2; π\PYZhy{}polarized light}
          \PY{n}{m} \PY{o}{=} \PY{l+m+mi}{133}\PY{o}{*}\PY{n}{amu}\PY{p}{;} \PY{c}{\PYZsh{} kg}
          \PY{n}{λ} \PY{o}{=} \PY{l+m+mi}{895}\PY{n}{u}\PY{l+s}{\PYZdq{}}\PY{l+s}{n}\PY{l+s}{m}\PY{l+s}{\PYZdq{}}\PY{p}{;} \PY{c}{\PYZsh{} D1 transition wavelength}
          \PY{n}{ω0} \PY{o}{=} \PY{l+m+mi}{2}\PY{o}{*}\PY{n+nb}{π}\PY{o}{*}\PY{n}{c}\PY{o}{/}\PY{n}{λ}\PY{p}{;}
          \PY{n}{ωhfs} \PY{o}{=} \PY{l+m+mi}{2}\PY{o}{*}\PY{n+nb}{pi}\PY{o}{*}\PY{l+m+mf}{9.193}\PY{n}{u}\PY{l+s}{\PYZdq{}}\PY{l+s}{G}\PY{l+s}{H}\PY{l+s}{z}\PY{l+s}{\PYZdq{}}\PY{p}{;}\PY{c}{\PYZsh{} Hz}
\end{Verbatim}


    \subsubsection{Beam Characteristics}\label{beam-characteristics}

    \begin{Verbatim}[commandchars=\\\{\}]
{\color{incolor}In [{\color{incolor}157}]:} \PY{n}{ω₀} \PY{o}{=} \PY{l+m+mi}{1}\PY{n}{u}\PY{l+s}{\PYZdq{}}\PY{l+s}{μ}\PY{l+s}{m}\PY{l+s}{\PYZdq{}}\PY{p}{;} \PY{c}{\PYZsh{} cm, beam radius/waist}
          \PY{n}{s₀} \PY{o}{=} \PY{l+m+mi}{10}\PY{p}{;} \PY{c}{\PYZsh{} saturation parameter}
          \PY{n}{I₀} \PY{o}{=} \PY{n}{s₀}\PY{o}{*}\PY{n}{Isat}\PY{p}{;} \PY{c}{\PYZsh{} W/cm\PYZca{}2}
          \PY{n}{Δ} \PY{o}{=} \PY{n}{ωhfs}\PY{o}{/}\PY{l+m+mi}{2}\PY{p}{;}
\end{Verbatim}


    \begin{Verbatim}[commandchars=\\\{\}]
{\color{incolor}In [{\color{incolor}266}]:} \PY{k}{function} \PY{n}{Polarizability}\PY{p}{(}\PY{n}{ω}\PY{p}{)}
              \PY{n}{C} \PY{o}{=} \PY{l+m+mi}{6}\PY{o}{*}\PY{n+nb}{π}\PY{o}{*}\PY{n}{ϵ0}\PY{o}{*}\PY{n}{c}\PY{o}{\PYZca{}}\PY{l+m+mi}{3}\PY{p}{;}
              \PY{k}{return} \PY{n}{C}\PY{o}{*}\PY{p}{(}\PY{n}{Γ}\PY{o}{/}\PY{n}{ω0}\PY{o}{\PYZca{}}\PY{l+m+mi}{2}\PY{p}{)}\PY{o}{/}\PY{p}{(}\PY{n}{ω0}\PY{o}{\PYZca{}}\PY{l+m+mi}{2}\PY{o}{\PYZhy{}}\PY{n}{ω}\PY{o}{\PYZca{}}\PY{l+m+mi}{2}\PY{o}{\PYZhy{}}\PY{n+nb}{im}\PY{o}{*}\PY{p}{(}\PY{n}{ω}\PY{o}{\PYZca{}}\PY{l+m+mi}{3}\PY{o}{/}\PY{n}{ω0}\PY{o}{\PYZca{}}\PY{l+m+mi}{2}\PY{p}{)}\PY{o}{*}\PY{n}{Γ}\PY{p}{)}
          \PY{k}{end}
          \PY{k}{function} \PY{n}{DipPotential}\PY{p}{(}\PY{n}{ω}\PY{p}{,}\PY{n+nb}{I}\PY{p}{)}
              \PY{n}{A} \PY{o}{=} \PY{o}{\PYZhy{}}\PY{l+m+mi}{1}\PY{o}{/}\PY{p}{(}\PY{l+m+mi}{2}\PY{o}{*}\PY{n}{ϵ0}\PY{o}{*}\PY{n}{c}\PY{p}{)}\PY{p}{;}
              \PY{k}{return} \PY{n}{A}\PY{o}{*}\PY{n}{real}\PY{p}{(}\PY{n}{Polarizability}\PY{p}{(}\PY{n}{ω}\PY{p}{)}\PY{p}{)}\PY{o}{*}\PY{n+nb}{I}\PY{p}{;}
          \PY{k}{end}
          \PY{k}{function} \PY{n}{GaussianIntensity}\PY{p}{(}\PY{n}{r}\PY{p}{,}\PY{n}{ω₀}\PY{p}{,}\PY{n}{I₀}\PY{p}{)}
              \PY{k}{return} \PY{n}{I₀}\PY{o}{*}\PY{n}{exp}\PY{o}{.}\PY{p}{(}\PY{o}{\PYZhy{}}\PY{l+m+mi}{2}\PY{o}{*}\PY{n}{r}\PY{o}{.\PYZca{}}\PY{l+m+mi}{2}\PY{o}{/}\PY{n}{ω₀}\PY{o}{\PYZca{}}\PY{l+m+mi}{2}\PY{p}{)}\PY{p}{;}
          \PY{k}{end}
          \PY{k}{function} \PY{n}{DipForce}\PY{p}{(}\PY{n}{r}\PY{p}{,}\PY{n}{ω}\PY{p}{,}\PY{n}{ω₀}\PY{p}{,}\PY{n}{I₀}\PY{p}{)}
              \PY{n}{U₀} \PY{o}{=} \PY{n}{DipPotential}\PY{p}{(}\PY{n}{ω}\PY{p}{,}\PY{n}{I₀}\PY{p}{)}\PY{p}{;}
              \PY{n}{B} \PY{o}{=} \PY{o}{\PYZhy{}}\PY{l+m+mi}{4}\PY{o}{*}\PY{n}{r}\PY{o}{/}\PY{n}{ω₀}\PY{o}{\PYZca{}}\PY{l+m+mi}{2}\PY{p}{;}
              \PY{k}{return} \PY{n}{U₀}\PY{o}{*}\PY{n}{B}\PY{o}{.*}\PY{n}{exp}\PY{o}{.}\PY{p}{(}\PY{o}{\PYZhy{}}\PY{l+m+mi}{2}\PY{o}{*}\PY{n}{r}\PY{o}{.\PYZca{}}\PY{l+m+mi}{2}\PY{o}{/}\PY{n}{ω₀}\PY{o}{\PYZca{}}\PY{l+m+mi}{2}\PY{p}{)}
          \PY{k}{end}
\end{Verbatim}


\begin{Verbatim}[commandchars=\\\{\}]
{\color{outcolor}Out[{\color{outcolor}266}]:} DipForce (generic function with 2 methods)
\end{Verbatim}
            
    \begin{Verbatim}[commandchars=\\\{\}]
{\color{incolor}In [{\color{incolor}262}]:} \PY{n}{r} \PY{o}{=} \PY{p}{[}\PY{o}{\PYZhy{}}\PY{l+m+mi}{3}\PY{o}{:}\PY{l+m+mf}{0.1}\PY{o}{:}\PY{l+m+mi}{3}\PY{p}{;}\PY{p}{]}\PY{o}{.*}\PY{n}{u}\PY{l+s}{\PYZdq{}}\PY{l+s}{μ}\PY{l+s}{m}\PY{l+s}{\PYZdq{}}\PY{p}{;}
          \PY{n}{F} \PY{o}{=} \PY{n}{DipForce}\PY{p}{(}\PY{n}{r}\PY{p}{,}\PY{n}{ω0}\PY{o}{\PYZhy{}}\PY{n}{Δ}\PY{p}{,}\PY{n}{ω₀}\PY{p}{,}\PY{n}{I₀}\PY{p}{)}\PY{p}{;}
          \PY{n}{a} \PY{o}{=} \PY{n}{upreferred}\PY{o}{.}\PY{p}{(}\PY{n}{F}\PY{p}{)}\PY{o}{/}\PY{n}{m}\PY{p}{;}
\end{Verbatim}


    \begin{Verbatim}[commandchars=\\\{\}]
{\color{incolor}In [{\color{incolor}274}]:} \PY{n}{figure}\PY{p}{;}
          \PY{n}{plot}\PY{p}{(}\PY{n}{ustrip}\PY{p}{(}\PY{n}{r}\PY{p}{)}\PY{p}{,}\PY{n}{ustrip}\PY{p}{(}\PY{n}{a}\PY{p}{)}\PY{p}{)}
          \PY{c}{\PYZsh{}vlines([\PYZhy{}1,1],\PYZhy{}3e12,3e12)}
          \PY{n}{ylabel}\PY{p}{(}\PY{l+s}{\PYZdq{}}\PY{l+s}{A}\PY{l+s}{c}\PY{l+s}{c}\PY{l+s}{e}\PY{l+s}{l}\PY{l+s}{e}\PY{l+s}{r}\PY{l+s}{a}\PY{l+s}{t}\PY{l+s}{i}\PY{l+s}{o}\PY{l+s}{n}\PY{l+s}{ }\PY{l+s}{[}\PY{l+s}{m}\PY{l+s}{/}\PY{l+s}{s}\PY{l+s}{²}\PY{l+s}{]}\PY{l+s}{\PYZdq{}}\PY{p}{)}
          \PY{n}{xlabel}\PY{p}{(}\PY{l+s}{\PYZdq{}}\PY{l+s}{P}\PY{l+s}{o}\PY{l+s}{s}\PY{l+s}{i}\PY{l+s}{t}\PY{l+s}{i}\PY{l+s}{o}\PY{l+s}{n}\PY{l+s}{ }\PY{l+s}{[}\PY{l+s}{μ}\PY{l+s}{m}\PY{l+s}{]}\PY{l+s}{\PYZdq{}}\PY{p}{)}\PY{p}{;}
          \PY{n}{plot}\PY{p}{(}\PY{n}{ustrip}\PY{p}{(}\PY{n}{r}\PY{p}{)}\PY{p}{,}\PY{l+m+mi}{3}\PY{o}{*}\PY{n}{ustrip}\PY{p}{(}\PY{n}{GaussianIntensity}\PY{p}{(}\PY{n}{r}\PY{p}{,}\PY{n}{ω₀}\PY{p}{,}\PY{n}{I₀}\PY{p}{)}\PY{p}{)}\PY{p}{)}\PY{p}{;}
          \PY{n}{legend}\PY{p}{(}\PY{p}{[}\PY{l+s}{\PYZdq{}}\PY{l+s}{A}\PY{l+s}{c}\PY{l+s}{c}\PY{l+s}{e}\PY{l+s}{l}\PY{l+s}{e}\PY{l+s}{r}\PY{l+s}{a}\PY{l+s}{t}\PY{l+s}{i}\PY{l+s}{o}\PY{l+s}{n}\PY{l+s}{\PYZdq{}}\PY{p}{,}\PY{l+s}{\PYZdq{}}\PY{l+s}{B}\PY{l+s}{e}\PY{l+s}{a}\PY{l+s}{m}\PY{l+s}{ }\PY{l+s}{I}\PY{l+s}{n}\PY{l+s}{t}\PY{l+s}{e}\PY{l+s}{n}\PY{l+s}{s}\PY{l+s}{i}\PY{l+s}{t}\PY{l+s}{y}\PY{l+s}{\PYZdq{}}\PY{p}{]}\PY{p}{)}\PY{p}{;}
\end{Verbatim}


    \begin{center}
    \adjustimage{max size={0.9\linewidth}{0.9\paperheight}}{output_12_0.png}
    \end{center}
    { \hspace*{\fill} \\}
    
    From the graph above we see that the direction of the kick is dependent
on which side of the beam center the atom is on. Also, in order to
experience a kick the atom has to be within the beam profile as
expected. The maximum kick occurs at
\(r_{max} = \pm\frac{\omega_0}{2}\).

    Knowing the maximum acceleration, how long should we wait to ensure we
can drive atom back toward the center?

    \subsubsection{Kinematic Eqautions}\label{kinematic-eqautions}

    \begin{Verbatim}[commandchars=\\\{\}]
{\color{incolor}In [{\color{incolor}293}]:} \PY{n}{t} \PY{o}{=} \PY{p}{[}\PY{l+m+mi}{0}\PY{o}{:}\PY{l+m+mf}{0.01}\PY{o}{:}\PY{l+m+mi}{200}\PY{p}{;}\PY{p}{]}\PY{o}{.*}\PY{n}{u}\PY{l+s}{\PYZdq{}}\PY{l+s}{μ}\PY{l+s}{s}\PY{l+s}{\PYZdq{}}\PY{p}{;}
          \PY{n}{amax} \PY{o}{=} \PY{n}{maximum}\PY{p}{(}\PY{n}{a}\PY{p}{)}\PY{p}{;}
          \PY{n}{x} \PY{o}{=} \PY{n}{upreferred}\PY{o}{.}\PY{p}{(}\PY{p}{(}\PY{l+m+mi}{1}\PY{o}{/}\PY{l+m+mi}{2}\PY{p}{)}\PY{o}{*}\PY{n}{amax}\PY{o}{.*}\PY{n}{t}\PY{o}{.\PYZca{}}\PY{l+m+mi}{2}\PY{p}{)}\PY{p}{;}
\end{Verbatim}


    \begin{Verbatim}[commandchars=\\\{\}]
{\color{incolor}In [{\color{incolor}302}]:} \PY{n}{figure}\PY{p}{;}
          \PY{n}{x} \PY{o}{=} \PY{n}{uconvert}\PY{o}{.}\PY{p}{(}\PY{n}{u}\PY{l+s}{\PYZdq{}}\PY{l+s}{μ}\PY{l+s}{m}\PY{l+s}{\PYZdq{}}\PY{p}{,}\PY{n}{x}\PY{p}{)}
          \PY{n}{plot}\PY{p}{(}\PY{n}{ustrip}\PY{p}{(}\PY{n}{t}\PY{p}{)}\PY{p}{,}\PY{n}{ustrip}\PY{p}{(}\PY{n}{x}\PY{p}{)}\PY{p}{)}
          \PY{n}{ylabel}\PY{p}{(}\PY{l+s}{\PYZdq{}}\PY{l+s}{P}\PY{l+s}{o}\PY{l+s}{s}\PY{l+s}{i}\PY{l+s}{t}\PY{l+s}{i}\PY{l+s}{o}\PY{l+s}{n}\PY{l+s}{ }\PY{l+s}{[}\PY{l+s}{μ}\PY{l+s}{m}\PY{l+s}{]}\PY{l+s}{\PYZdq{}}\PY{p}{)}
          \PY{n}{xlabel}\PY{p}{(}\PY{l+s}{\PYZdq{}}\PY{l+s}{T}\PY{l+s}{i}\PY{l+s}{m}\PY{l+s}{e}\PY{l+s}{ }\PY{l+s}{[}\PY{l+s}{μ}\PY{l+s}{s}\PY{l+s}{]}\PY{l+s}{\PYZdq{}}\PY{p}{)}\PY{p}{;}
          \PY{c}{\PYZsh{}hlines(ustrip(ω₀)/2,0,200)}
\end{Verbatim}


    \begin{center}
    \adjustimage{max size={0.9\linewidth}{0.9\paperheight}}{output_17_0.png}
    \end{center}
    { \hspace*{\fill} \\}
    
    The above plot assumes the atom starts off at the maximum location to
experience the largest acceleration. It then proceeds to accelerate and
change position. In this case it starts off at the position of
\(\omega_0/2\) and thus is constrained to move a maximum distance of
\(\omega_0/2\) before it leaves the beam intensity region.

    Our current capabilities. 1. microwave \(\pi\) pulse: 100-200 \(\mu s\)


    % Add a bibliography block to the postdoc
    
    
    
    \end{document}
